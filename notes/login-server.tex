\documentclass{article}
\usepackage[T1]{fontenc}
\usepackage{lmodern}

\usepackage{graphicx}
\usepackage[margin=1in]{geometry}
\usepackage{float}
\usepackage{xcolor}
\usepackage{hyperref}
\usepackage{float}
\usepackage{amsmath}

\begin{document}

\title{PSForever Server Notes}

\author{Chord $<$chord@tuta.io$>$}

\maketitle

%\section*{Security Model}
%
%\subsection*{Trust Model}
%
%\subsection*{Threat Model}
%
%\subsection*{Security Objects}
%
%\subsubsection*{Client Objects}

\section*{FAQ}
\paragraph*{A new packet comes in, who is it for?}
A UDP packet has a source address and port. This pair will allow the
server to identify which session the packet is for. If no sessions are found
matching this pair, then a new session is created. Essentially packets have to
be routed to session objects which know what to do with it, given the state of the session.

\paragraph*{What is a session and what type of information does a session store?}
A session is some state used to represent a unique user's connection. This
state can be the current connection's status, the last time a message was
received for this session, and much more.

\paragraph*{}

%\section*{Code Structure}

%%%%%%%%%%%%%%%%%%%%%%%%%%%%%%%%%%%%%%
% Begin References
%%%%%%%%%%%%%%%%%%%%%%%%%%%%%%%%%%%%%%
\begin{thebibliography}{99}

\end{thebibliography}

\end{document}

\end{document}
